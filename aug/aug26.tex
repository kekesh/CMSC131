\section{Monday, August 26, 2019}

\subsection{Logistics}

This is CMSC 131: Object Oriented Programming I. This course is an introduction to Java, and it does not assume any programming knowledge. 

\begin{itemize}
    \item The course homepage is at \url{https://www.cs.umd.edu/class/fall2019/cmsc131-030X/}.
    \item Course announcements are sent out through Piazza.
    \item Projects are worth $26\%$ of our grade, quizzes and exercises are worth $16\%$, the three midterms are worth $30\%$, and the final exam is worth $28\%$.
    \item All projects are due at $11:30$ p.m. on the specified day in the project description. However, you can submit up to $24$ hours afterwards with a $12\%$ penalty.
    \item If you submit a project multiple times, the highest scoring project gets graded.
    \item All lectures are recorded and posted to Panopto.
\end{itemize}


\subsection{Preliminaries}

We'll start this course off by introducing some important terminology. \\

Firstly, we'll briefly discuss two levels of software:  
\begin{enumerate}
    \item \vocab{Operating systems} manage the computer's resources; they are typically run as soon as a computer is turned on. Some examples include security-related software, and process management tools.
    \item \vocab{Applications} are programs that users interact directly with. These are typically explicitly run by the user. This can include word processors, games, music software, or java programs.
\end{enumerate}


Programs are typically executed with the help of \vocab{compilers}. Compilers are programs used for translating other programs (``source code") that you write into assembler or machine code. There are many compilers out there, but we only need one. An alternative way to execute programs is through the use of \vocab{interpreters}, which take source code as input and execute the source directly. However, these are much slower than compiled programs. \vocab{Debuggers} are based on interpreters since they support the step-by-step execution of source code.
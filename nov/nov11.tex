\section{Monday, November 11, 2019}


Today, we'll continue talking about interfaces.

As a brief recap, recall that interfaces define a ``contract" between the classes that implement it: the classes that implement an interface are \textit{required} to provide implementations of the methods specified by the interface. Furthermore, interfaces define an ``is-a" relationship between their classes. If we have an interface \verb!A! and a class \verb!B! implements \verb!A!, then we can treat \verb!B! as if it had type \verb!A! This allows us to classify \textit{everything} that implements an interface as a single entity (namely, by the interface itself).


\subsection{The Comparable Interface}

The \vocab{comparable interface} is an interface provided by Java that can be used in order to provide an ordering to objects of a user-defined class. When a class implements this interface, it must implement the \verb!compareTo()! method, which returns a negative value if the current object is less than the object being compared to, zero if they're equal, and positive otherwise (if we don't implement this method, then our code won't compile!).

As an example, consider the following code segment:

\begin{lstlisting}
package equalsComparable;
import java.util.ArrayList;

public class Example {

    public static void main(String[] args) {
        ArrayList<String> l = new ArrayList<String>();
        l.add("ice_cream");
        l.add("apple");
        l.add("raw_onion");
    }
}
\end{lstlisting}


If we were to print this ArrayList with \verb!System.out.println(l)!, we'd get the output \verb![ice_cream, apple, raw_onion]!, which is the order in which we inserted the elements into the ArrayList.


But now, what if we wanted to sort our elements and then print them? Calling \verb!Collections.sort(l)! and then printing would sort our items lexicographically, so our output would be \verb![apple, ice_cream, raw_onion]!. Why does this happen? Because the string class has a built-in \verb!compareTo! method, which sorts lexicographically. 


It turns out that we can define our own \verb!compareTo()! method and sort our elements according to how we want. As mentioned previously, this can be done by implementing the \verb!Comparable! interface. 


Consider the following  code segment, which illustrates how we can use the \verb!Comparable! interface:



\begin{lstlisting}
public class Student implements Comparable<Student> {
	private String name;
	private int id;

	public Student(String name, int id) {
		this.name = name;
		this.id = id;
	}

	public String toString() {
		return "Name: " + name + ", Id: " + id;
	}

	public int compareTo(Student other) {
		if (id < other.id) {
			return -1;
		} else if (id == other.id) {
			return 0;
		} else {
			return 1;
		}
	}

	public boolean equals(Object obj) {
		if (obj == this) {
			return true;
		} else if (!(obj instanceof Student)) {
			return false;
		} else {
			return compareTo((Student) obj) == 0;
		}
	}

	/* If we override equals we must have correct hashCode */
	public int hashCode() {
		return id;
	}
}
\end{lstlisting}

Note that the \verb!Student! class implements the \verb!Comparable! interface (whatever comes in the angled brackets \verb!< ... >! specifies what we will be comparing). Consequently, the  \verb!compareTo! method is overriden so that we can specify that we want to sort the students based on their IDs (we retunr a negative value when the student's ID is less than the other students, zero when they're equal, and a positive value otherwise). Now, if we use \verb!Collections.sort()! on an ArrayList of \verb!Student! objects, we will sort based on ID. 

An even simpler implementation of \verb!compareTo! that would do just what we want would be to only have the statement \verb!return id - other.id!.

Finally, we have a \verb!hashCode! method, which will be discussed in-depth at a later time. But for now, it's important to know that, whenever the \verb!equals! method is overriden, we must provide a \verb!hashCode! method with the same header as shown in the code segment above. Java's enforcement of this ``policy" is known as a \vocab{hash code contract}. However, we won't discuss the hash code contract any further until CMSC 132.

\subsection{Polymorphism}

Using an interface, we can create a variable that can reference objects of different types (like we saw in a previous lecture, we could use an \verb!Animal! variable to refer to objects whose types were \verb!Dog! and \verb!Piranha!). This form of ``generalization" is known as \vocab{polymorphism}, and it is one of the hallmarks of object-oriented programming. 

\subsection{Wrappers}

A \vocab{primitive data type} in Java is a basic data type provided to us as a ``building block". For example, \verb!byte, short, int, long, double, char!, and \verb!boolean! are all primitive types. 

Each primitive type also has a corresponding \vocab{wrapper class}, which is used to ``wrap" a class around a primitive type (for example, \verb!int!'s wrapper class is \verb!Integer!, \verb!double!'s wrapper class is \verb!Double!, and \verb!char!'s wrapper class is \verb!Character!). 

Why do we need wrapper classes? Because primitive types themselves are not objects. Wrapper classes are used in order to convert data types to objects in case this is ever necessary. 

Each wrapper provides a number of useful methods:

\begin{itemize}
    \item The \verb!Integer! wrapper's constructor can be used to create a new \verb!Integer! object by writing something like, \verb!Integer x = new Integer(123)!.
    \item The \verb!Integer! wrapper class defines static constant variables named \verb!MAX_VALUE! and \verb!MIN_VALUE!, which can be used to check the largest and smallest values that can be stored in an integer variable. We can access these constants by writing \verb!Integer.MAX_VALUE! and \verb!Integer.MIN_VALUE!. 
    \item The \verb!Integer! class also implements a \verb!parseInt! method, which can be used to convert a string to an integer. For example, one might write \verb!int k = Integer.parseInt("123");! in order to store the value $123$ in \verb!k!.
\end{itemize}

The following code segment illustrates some other useful methods and constants that the wrapper classes provide to us:

\begin{lstlisting}
package examples;

public class WrapperExample {
	public static void main(String[] args) {
		Integer x = new Integer(10);
		Integer y = 20; // Notice: no need for new
		int w = x - 100;

		int comparison = x.compareTo(y);
		System.out.println("Comparison result: " + comparison);
		comparison = x.compareTo(w);
		System.out.println("Comparison result: " + comparison);

		System.out.println("MAX_VALUE: " + Integer.MAX_VALUE);
		System.out.println("MIN_VALUE: " + Integer.MIN_VALUE);

		/* Compare the next value with the previous one */
		System.out.println("Abs(MIN_VALUE): " + Math.abs(Integer.MIN_VALUE));

		System.out.println("NEGATIVE_INFINITY: " + Double.NEGATIVE_INFINITY);
		System.out.println("POSITIVE_INFINITY: " + Double.POSITIVE_INFINITY);

		System.out.println("35 in binary: " + Integer.toBinaryString(35));
		System.out.println("35 in hex: " + Integer.toHexString(35));

		Character c = 'A';
		System.out.println("Character: " + c);

		Boolean b = true;
		System.out.println("Boolean: " + b);
	}
}
\end{lstlisting}

\subsection{Method Overloading}

Next, we'll discuss \vocab{method overloading}. 

Java allows methods to have the same name, even within the same class. We've seen this before --- many of our classes have had several constructors, which have the same names. As another example, suppose we're given a \verb!Date! class. We can add the following methods to change the current dates (implementations are omitted):

\begin{lstlisting}
public void setDate(int m, int d, int y) { ... } // month given as integer

public void setDate(String m, int d, int y) { .. } // month given as string

public void setDate(int m, int y) { ... } /* automatically set day to 1 */
\end{lstlisting}

This is clearly useful since we don't know what information an end-user might provide us with. Method overloading is typically performed when the overloaded method performs a very similar function to the original method, but we want to provide the user with different ways of calling the method. \\

As an example, suppose we want to implement a \verb!max! method that returns the maximum of two parameters provided by the user. If we wanted the user to be able to provide either doubles or integers, then we can overload the \verb!max! method to take integers or doubles as the parameter. Without method overloading, we'd need to create two methods with distinct names (e.g. \verb!maxIntegers! and \verb!maxDoubles!), and this could quickly get tedious for our user.

How does Java know which function to call? It looks at the number of arguments that the user provides. If the number of arguments is the same for two different methods with the same name, then Java goes on to look at the type of the arguments provided by the user. 